% Options for packages loaded elsewhere
\PassOptionsToPackage{unicode}{hyperref}
\PassOptionsToPackage{hyphens}{url}
%
\documentclass[
]{article}
\usepackage{amsmath,amssymb}
\usepackage{lmodern}
\usepackage{ifxetex,ifluatex}
\ifnum 0\ifxetex 1\fi\ifluatex 1\fi=0 % if pdftex
  \usepackage[T1]{fontenc}
  \usepackage[utf8]{inputenc}
  \usepackage{textcomp} % provide euro and other symbols
\else % if luatex or xetex
  \usepackage{unicode-math}
  \defaultfontfeatures{Scale=MatchLowercase}
  \defaultfontfeatures[\rmfamily]{Ligatures=TeX,Scale=1}
\fi
% Use upquote if available, for straight quotes in verbatim environments
\IfFileExists{upquote.sty}{\usepackage{upquote}}{}
\IfFileExists{microtype.sty}{% use microtype if available
  \usepackage[]{microtype}
  \UseMicrotypeSet[protrusion]{basicmath} % disable protrusion for tt fonts
}{}
\makeatletter
\@ifundefined{KOMAClassName}{% if non-KOMA class
  \IfFileExists{parskip.sty}{%
    \usepackage{parskip}
  }{% else
    \setlength{\parindent}{0pt}
    \setlength{\parskip}{6pt plus 2pt minus 1pt}}
}{% if KOMA class
  \KOMAoptions{parskip=half}}
\makeatother
\usepackage{xcolor}
\IfFileExists{xurl.sty}{\usepackage{xurl}}{} % add URL line breaks if available
\IfFileExists{bookmark.sty}{\usepackage{bookmark}}{\usepackage{hyperref}}
\hypersetup{
  pdftitle={Analysis: insecta diversity data in China and Brazil},
  pdfauthor={Lilei Chen},
  hidelinks,
  pdfcreator={LaTeX via pandoc}}
\urlstyle{same} % disable monospaced font for URLs
\usepackage[margin=1in]{geometry}
\usepackage{color}
\usepackage{fancyvrb}
\newcommand{\VerbBar}{|}
\newcommand{\VERB}{\Verb[commandchars=\\\{\}]}
\DefineVerbatimEnvironment{Highlighting}{Verbatim}{commandchars=\\\{\}}
% Add ',fontsize=\small' for more characters per line
\usepackage{framed}
\definecolor{shadecolor}{RGB}{248,248,248}
\newenvironment{Shaded}{\begin{snugshade}}{\end{snugshade}}
\newcommand{\AlertTok}[1]{\textcolor[rgb]{0.94,0.16,0.16}{#1}}
\newcommand{\AnnotationTok}[1]{\textcolor[rgb]{0.56,0.35,0.01}{\textbf{\textit{#1}}}}
\newcommand{\AttributeTok}[1]{\textcolor[rgb]{0.77,0.63,0.00}{#1}}
\newcommand{\BaseNTok}[1]{\textcolor[rgb]{0.00,0.00,0.81}{#1}}
\newcommand{\BuiltInTok}[1]{#1}
\newcommand{\CharTok}[1]{\textcolor[rgb]{0.31,0.60,0.02}{#1}}
\newcommand{\CommentTok}[1]{\textcolor[rgb]{0.56,0.35,0.01}{\textit{#1}}}
\newcommand{\CommentVarTok}[1]{\textcolor[rgb]{0.56,0.35,0.01}{\textbf{\textit{#1}}}}
\newcommand{\ConstantTok}[1]{\textcolor[rgb]{0.00,0.00,0.00}{#1}}
\newcommand{\ControlFlowTok}[1]{\textcolor[rgb]{0.13,0.29,0.53}{\textbf{#1}}}
\newcommand{\DataTypeTok}[1]{\textcolor[rgb]{0.13,0.29,0.53}{#1}}
\newcommand{\DecValTok}[1]{\textcolor[rgb]{0.00,0.00,0.81}{#1}}
\newcommand{\DocumentationTok}[1]{\textcolor[rgb]{0.56,0.35,0.01}{\textbf{\textit{#1}}}}
\newcommand{\ErrorTok}[1]{\textcolor[rgb]{0.64,0.00,0.00}{\textbf{#1}}}
\newcommand{\ExtensionTok}[1]{#1}
\newcommand{\FloatTok}[1]{\textcolor[rgb]{0.00,0.00,0.81}{#1}}
\newcommand{\FunctionTok}[1]{\textcolor[rgb]{0.00,0.00,0.00}{#1}}
\newcommand{\ImportTok}[1]{#1}
\newcommand{\InformationTok}[1]{\textcolor[rgb]{0.56,0.35,0.01}{\textbf{\textit{#1}}}}
\newcommand{\KeywordTok}[1]{\textcolor[rgb]{0.13,0.29,0.53}{\textbf{#1}}}
\newcommand{\NormalTok}[1]{#1}
\newcommand{\OperatorTok}[1]{\textcolor[rgb]{0.81,0.36,0.00}{\textbf{#1}}}
\newcommand{\OtherTok}[1]{\textcolor[rgb]{0.56,0.35,0.01}{#1}}
\newcommand{\PreprocessorTok}[1]{\textcolor[rgb]{0.56,0.35,0.01}{\textit{#1}}}
\newcommand{\RegionMarkerTok}[1]{#1}
\newcommand{\SpecialCharTok}[1]{\textcolor[rgb]{0.00,0.00,0.00}{#1}}
\newcommand{\SpecialStringTok}[1]{\textcolor[rgb]{0.31,0.60,0.02}{#1}}
\newcommand{\StringTok}[1]{\textcolor[rgb]{0.31,0.60,0.02}{#1}}
\newcommand{\VariableTok}[1]{\textcolor[rgb]{0.00,0.00,0.00}{#1}}
\newcommand{\VerbatimStringTok}[1]{\textcolor[rgb]{0.31,0.60,0.02}{#1}}
\newcommand{\WarningTok}[1]{\textcolor[rgb]{0.56,0.35,0.01}{\textbf{\textit{#1}}}}
\usepackage{graphicx}
\makeatletter
\def\maxwidth{\ifdim\Gin@nat@width>\linewidth\linewidth\else\Gin@nat@width\fi}
\def\maxheight{\ifdim\Gin@nat@height>\textheight\textheight\else\Gin@nat@height\fi}
\makeatother
% Scale images if necessary, so that they will not overflow the page
% margins by default, and it is still possible to overwrite the defaults
% using explicit options in \includegraphics[width, height, ...]{}
\setkeys{Gin}{width=\maxwidth,height=\maxheight,keepaspectratio}
% Set default figure placement to htbp
\makeatletter
\def\fps@figure{htbp}
\makeatother
\setlength{\emergencystretch}{3em} % prevent overfull lines
\providecommand{\tightlist}{%
  \setlength{\itemsep}{0pt}\setlength{\parskip}{0pt}}
\setcounter{secnumdepth}{-\maxdimen} % remove section numbering
\ifluatex
  \usepackage{selnolig}  % disable illegal ligatures
\fi

\title{Analysis: insecta diversity data in China and Brazil}
\author{Lilei Chen}
\date{2021/8/5}

\begin{document}
\maketitle

\hypertarget{load-the-packages}{%
\subsection{Load the packages}\label{load-the-packages}}

This R markdown included every step I have done. I changed the
categories to suit a more reasonable analysis, I have also merged sites
to deal with the situations such as ten traps for same land use, use
intensity, coordinates and start/end times are treated as 10 sites,
which is better to treat them as one. Code are derived from De
Palma(2019), a work-through about PREDICTS data.

Load the packages we are going to use.

\begin{Shaded}
\begin{Highlighting}[]
\FunctionTok{library}\NormalTok{(dplyr) }\CommentTok{\# for easy data manipulation}
\end{Highlighting}
\end{Shaded}

\begin{verbatim}
## Warning: package 'dplyr' was built under R version 4.0.5
\end{verbatim}

\begin{verbatim}
## 
## Attaching package: 'dplyr'
\end{verbatim}

\begin{verbatim}
## The following objects are masked from 'package:stats':
## 
##     filter, lag
\end{verbatim}

\begin{verbatim}
## The following objects are masked from 'package:base':
## 
##     intersect, setdiff, setequal, union
\end{verbatim}

\begin{Shaded}
\begin{Highlighting}[]
\FunctionTok{library}\NormalTok{(tidyr) }\CommentTok{\# ditto}
\end{Highlighting}
\end{Shaded}

\begin{verbatim}
## Warning: package 'tidyr' was built under R version 4.0.5
\end{verbatim}

\begin{Shaded}
\begin{Highlighting}[]
\FunctionTok{library}\NormalTok{(lme4) }\CommentTok{\# for mixed effects models}
\end{Highlighting}
\end{Shaded}

\begin{verbatim}
## Warning: package 'lme4' was built under R version 4.0.5
\end{verbatim}

\begin{verbatim}
## Loading required package: Matrix
\end{verbatim}

\begin{verbatim}
## 
## Attaching package: 'Matrix'
\end{verbatim}

\begin{verbatim}
## The following objects are masked from 'package:tidyr':
## 
##     expand, pack, unpack
\end{verbatim}

\begin{Shaded}
\begin{Highlighting}[]
\FunctionTok{library}\NormalTok{(sjPlot) }\CommentTok{\#for visualizing results}
\end{Highlighting}
\end{Shaded}

\begin{verbatim}
## Warning: package 'sjPlot' was built under R version 4.0.5
\end{verbatim}

\begin{verbatim}
## Registered S3 methods overwritten by 'parameters':
##   method                           from      
##   as.double.parameters_kurtosis    datawizard
##   as.double.parameters_skewness    datawizard
##   as.double.parameters_smoothness  datawizard
##   as.numeric.parameters_kurtosis   datawizard
##   as.numeric.parameters_skewness   datawizard
##   as.numeric.parameters_smoothness datawizard
##   print.parameters_distribution    datawizard
##   print.parameters_kurtosis        datawizard
##   print.parameters_skewness        datawizard
##   summary.parameters_kurtosis      datawizard
##   summary.parameters_skewness      datawizard
\end{verbatim}

\begin{Shaded}
\begin{Highlighting}[]
\FunctionTok{library}\NormalTok{(car) }\CommentTok{\# for getting anova tables with significance values}
\end{Highlighting}
\end{Shaded}

\begin{verbatim}
## Warning: package 'car' was built under R version 4.0.5
\end{verbatim}

\begin{verbatim}
## Loading required package: carData
\end{verbatim}

\begin{verbatim}
## 
## Attaching package: 'car'
\end{verbatim}

\begin{verbatim}
## The following object is masked from 'package:dplyr':
## 
##     recode
\end{verbatim}

\begin{Shaded}
\begin{Highlighting}[]
\FunctionTok{library}\NormalTok{(ggplot2) }\CommentTok{\# for plotting}
\end{Highlighting}
\end{Shaded}

\begin{verbatim}
## Warning: package 'ggplot2' was built under R version 4.0.5
\end{verbatim}

\begin{Shaded}
\begin{Highlighting}[]
\FunctionTok{library}\NormalTok{(DHARMa) }\CommentTok{\# for model criticism plots}
\end{Highlighting}
\end{Shaded}

\begin{verbatim}
## Warning: package 'DHARMa' was built under R version 4.0.5
\end{verbatim}

\begin{verbatim}
## This is DHARMa 0.4.3. For overview type '?DHARMa'. For recent changes, type news(package = 'DHARMa') Note: Syntax of plotResiduals has changed in 0.3.0, see ?plotResiduals for details
\end{verbatim}

\begin{Shaded}
\begin{Highlighting}[]
\FunctionTok{library}\NormalTok{(effects) }\CommentTok{\# for extracting model effects}
\end{Highlighting}
\end{Shaded}

\begin{verbatim}
## Warning: package 'effects' was built under R version 4.0.5
\end{verbatim}

\begin{verbatim}
## lattice theme set by effectsTheme()
## See ?effectsTheme for details.
\end{verbatim}

\begin{Shaded}
\begin{Highlighting}[]
\FunctionTok{library}\NormalTok{(merTools) }\CommentTok{\# using it for extracting estimates for plotting}
\end{Highlighting}
\end{Shaded}

\begin{verbatim}
## Warning: package 'merTools' was built under R version 4.0.5
\end{verbatim}

\begin{verbatim}
## Loading required package: arm
\end{verbatim}

\begin{verbatim}
## Warning: package 'arm' was built under R version 4.0.5
\end{verbatim}

\begin{verbatim}
## Loading required package: MASS
\end{verbatim}

\begin{verbatim}
## 
## Attaching package: 'MASS'
\end{verbatim}

\begin{verbatim}
## The following object is masked from 'package:dplyr':
## 
##     select
\end{verbatim}

\begin{verbatim}
## 
## arm (Version 1.11-2, built: 2020-7-27)
\end{verbatim}

\begin{verbatim}
## Working directory is F:/Code_PREDICTS/PREDICTS
\end{verbatim}

\begin{verbatim}
## 
## Attaching package: 'arm'
\end{verbatim}

\begin{verbatim}
## The following object is masked from 'package:car':
## 
##     logit
\end{verbatim}

\hypertarget{read-in-and-process-the-diversity-data}{%
\subsection{Read in and process the diversity
data}\label{read-in-and-process-the-diversity-data}}

Read in the raw data from PREDICTS.

\begin{Shaded}
\begin{Highlighting}[]
\CommentTok{\# set working directory}
\FunctionTok{setwd}\NormalTok{(}\StringTok{"F:/Code\_PREDICTS/data"}\NormalTok{)}

\CommentTok{\# read rds file}
\NormalTok{diversity }\OtherTok{\textless{}{-}} \FunctionTok{readRDS}\NormalTok{(}\StringTok{"diversity{-}2021{-}08{-}01{-}02{-}33{-}01.rds"}\NormalTok{)}
\end{Highlighting}
\end{Shaded}

\hypertarget{insecta-data-in-china-and-brazil}{%
\subsection{Insecta data in China and
Brazil}\label{insecta-data-in-china-and-brazil}}

Now we have data from PREDICTS, this study focus on insecta in China and
Brazil, therefore, the next step is to select such data we want.

\begin{Shaded}
\begin{Highlighting}[]
\NormalTok{Insecta\_diversity }\OtherTok{\textless{}{-}} \FunctionTok{filter}\NormalTok{(diversity, Class }\SpecialCharTok{==} \StringTok{"Insecta"}\NormalTok{) }\SpecialCharTok{\%\textgreater{}\%}
                      \FunctionTok{filter}\NormalTok{(Country }\SpecialCharTok{==} \StringTok{"China"}\SpecialCharTok{|}\NormalTok{ Country }\SpecialCharTok{==} \StringTok{"Brazil"}\NormalTok{)}
\end{Highlighting}
\end{Shaded}

Let's have a look about the data to see if filter()worked.

\begin{Shaded}
\begin{Highlighting}[]
\FunctionTok{unique}\NormalTok{(Insecta\_diversity}\SpecialCharTok{$}\NormalTok{Country)}
\end{Highlighting}
\end{Shaded}

\begin{verbatim}
## [1] Brazil China 
## 246 Levels: Afghanistan 脜land Islands Albania Algeria ... Zimbabwe
\end{verbatim}

\hypertarget{land-use-and-use-intensity}{%
\subsection{Land use and use
intensity}\label{land-use-and-use-intensity}}

We are focusing on land use and use intensity, the columns of
\texttt{Predominant\_habitat} and \texttt{Use\_intensity} contain
information of land use and use intensity. let's check the details.

\begin{verbatim}
##        
##         Primary forest Primary non-forest Young secondary vegetation
##   FALSE          35048              34398                      41667
##   TRUE             298                 98                       1137
##        
##         Intermediate secondary vegetation Mature secondary vegetation
##   FALSE                             16937                       52231
##   TRUE                                704                           0
##        
##         Secondary vegetation (indeterminate age) Plantation forest Pasture
##   FALSE                                     2202             20111   46805
##   TRUE                                      1968               770     492
##        
##         Cropland Urban Cannot decide
##   FALSE    12431   258             0
##   TRUE      3116  6387          1418
\end{verbatim}

It seems that China has 0 mature secondary vegetation and fair less
numbers of primary forest and primary non-forest. We need to combine
some of these levels for further analysis.

\begin{verbatim}
##        
##         Minimal use Light use Intense use Cannot decide
##   FALSE      180167     78257        2832           832
##   TRUE         4128      5920        1616          4724
\end{verbatim}

Although the data of use intensity in China is less than data in Brazil,
it is able to conduct the analysis. Based on the data, We are going to
combine Primary forest and Primary non-forest as Primary vegetation.
Young secondary vegetation, Intermediate secondary vegetation, Mature
secondary vegetation and Secondary vegetation (indeterminate age) will
be combined as Secondary vegetation. The Cannot decide in Land use and
use intensity will be put as NA.

\begin{Shaded}
\begin{Highlighting}[]
\CommentTok{\# combine to primary vegetation and secondary vegetation}
\NormalTok{relevled\_diversity }\OtherTok{\textless{}{-}}\NormalTok{ Insecta\_diversity }\SpecialCharTok{\%\textgreater{}\%}
     \CommentTok{\# combine primary forest and primary non{-}forest into primary vegetation}
  \FunctionTok{mutate}\NormalTok{(}
    \AttributeTok{LandUse =} \FunctionTok{ifelse}\NormalTok{(Predominant\_habitat }\SpecialCharTok{==} \StringTok{"Primary forest"} \SpecialCharTok{|}
\NormalTok{                       Predominant\_habitat }\SpecialCharTok{==} \StringTok{"Primary non{-}forest"}\NormalTok{,}
                     \StringTok{"Primary vegetation"}\NormalTok{,}
                     \FunctionTok{paste}\NormalTok{(Predominant\_habitat)),}
    \CommentTok{\# combine all secondary vegetation}
    \AttributeTok{LandUse =} \FunctionTok{ifelse}\NormalTok{(Predominant\_habitat }\SpecialCharTok{==} \StringTok{"Young secondary vegetation"}\SpecialCharTok{|}
\NormalTok{                       Predominant\_habitat }\SpecialCharTok{==} \StringTok{"Intermediate secondary vegetation"}\SpecialCharTok{|}
\NormalTok{                       Predominant\_habitat }\SpecialCharTok{==} \StringTok{"Mature secondary vegetation"}\SpecialCharTok{|}
\NormalTok{                       Predominant\_habitat }\SpecialCharTok{==} \StringTok{"Secondary vegetation (indeterminate age)"}\NormalTok{,}
                     \StringTok{"Secondary vegetation"}\NormalTok{,}
                     \FunctionTok{paste}\NormalTok{(LandUse)),}
    
    \CommentTok{\# change cannot decide into NA}
    \AttributeTok{LandUse =} \FunctionTok{ifelse}\NormalTok{(Predominant\_habitat }\SpecialCharTok{==} \StringTok{"Cannot decide"}\NormalTok{,}
                     \ConstantTok{NA}\NormalTok{, }
                     \FunctionTok{paste}\NormalTok{(LandUse)),}
    \CommentTok{\# relevel the factor so that Primary vegetation will be the ref (so that it is the intercept term in models)}
    \AttributeTok{LandUse =} \FunctionTok{factor}\NormalTok{(LandUse),}
    \AttributeTok{LandUse =} \FunctionTok{relevel}\NormalTok{(LandUse, }\AttributeTok{ref =} \StringTok{"Primary vegetation"}\NormalTok{),}
    
    \CommentTok{\# change cannot decide into NA in UseIntensity}
    \AttributeTok{UseIntensity =} \FunctionTok{ifelse}\NormalTok{(Use\_intensity }\SpecialCharTok{==} \StringTok{"Cannot decide"}\NormalTok{,}
                          \ConstantTok{NA}\NormalTok{,}
                          \FunctionTok{paste}\NormalTok{(Use\_intensity)),}
    \CommentTok{\# relevel the factor so that Minimal use is the first level (so that it is the intercept term in models)}
    \AttributeTok{UseIntensity =} \FunctionTok{factor}\NormalTok{(UseIntensity),}
    \AttributeTok{UseIntensity =} \FunctionTok{relevel}\NormalTok{(UseIntensity, }\AttributeTok{ref =} \StringTok{"Minimal use"}\NormalTok{)}
\NormalTok{  )}

\CommentTok{\# check \textquotesingle{}LandUse\textquotesingle{} and \textquotesingle{}UseIntensity\textquotesingle{}}
\FunctionTok{table}\NormalTok{(relevled\_diversity}\SpecialCharTok{$}\NormalTok{LandUse)}
\end{Highlighting}
\end{Shaded}

\begin{verbatim}
## 
##   Primary vegetation             Cropland              Pasture 
##                69842                15547                47297 
##    Plantation forest Secondary vegetation                Urban 
##                20881               116846                 6645
\end{verbatim}

\begin{Shaded}
\begin{Highlighting}[]
\FunctionTok{table}\NormalTok{(relevled\_diversity}\SpecialCharTok{$}\NormalTok{UseIntensity)}
\end{Highlighting}
\end{Shaded}

\begin{verbatim}
## 
## Minimal use Intense use   Light use 
##      184295        4448       84177
\end{verbatim}

\hypertarget{correct-sampling-effort}{%
\subsection{Correct sampling effort}\label{correct-sampling-effort}}

Now we have releveled data with suitable categories for further
analysis. Next, we need abundance and richness data of insecta. To do
that, the rescale of sampling effort is necessary for diversity
measurements which are sensitive to sampling effort. We rescale sampling
effort for each study to have a maximum value of 1. Let's have a look if
there are studies with missing sampling effort.

\begin{Shaded}
\begin{Highlighting}[]
\NormalTok{studies\_check }\OtherTok{\textless{}{-}}\NormalTok{ relevled\_diversity }\SpecialCharTok{\%\textgreater{}\%}
  \CommentTok{\# filter the rows where Sampling efforts are NA}
  \FunctionTok{filter}\NormalTok{(}\FunctionTok{is.na}\NormalTok{(Sampling\_effort)) }\SpecialCharTok{\%\textgreater{}\%}
  \CommentTok{\# keep only unique studies}
  \FunctionTok{distinct}\NormalTok{(SS) }\SpecialCharTok{\%\textgreater{}\%}
  \CommentTok{\# pull the vector}
  \FunctionTok{pull}\NormalTok{(SS)}

\NormalTok{relevled\_diversity }\SpecialCharTok{\%\textgreater{}\%}
  \CommentTok{\# filter the rows where studies are those that had missing sampling efforts (above)}
  \FunctionTok{filter}\NormalTok{(SS }\SpecialCharTok{\%in\%}\NormalTok{ studies\_check) }\SpecialCharTok{\%\textgreater{}\%}
  \CommentTok{\# drop missing levels}
  \FunctionTok{droplevels}\NormalTok{() }\SpecialCharTok{\%\textgreater{}\%}
  \CommentTok{\# pull out the sampling efforts of these studies}
  \FunctionTok{pull}\NormalTok{(Sampling\_effort)}\SpecialCharTok{\%\textgreater{}\%}
  \CommentTok{\# summaries to check that ALL the data are NAs}
  \FunctionTok{summary}\NormalTok{() }
\end{Highlighting}
\end{Shaded}

\begin{verbatim}
##    Min. 1st Qu.  Median    Mean 3rd Qu.    Max. 
## 
\end{verbatim}

There is no NA for sampling effort of all studies so we don't need to
replace any value. Otherwise we should replace NAs as 1 assuming that
the sampling efforts don't vary within a study and correct those that
need it.

\begin{Shaded}
\begin{Highlighting}[]
\NormalTok{relevled\_diversity }\OtherTok{\textless{}{-}}\NormalTok{ relevled\_diversity }\SpecialCharTok{\%\textgreater{}\%}
  
  \CommentTok{\# group by Study}
  \FunctionTok{group\_by}\NormalTok{(SS) }\SpecialCharTok{\%\textgreater{}\%}
  
  \CommentTok{\# check how many sampling efforts there are in each study}
  \FunctionTok{mutate}\NormalTok{(}\AttributeTok{n\_sample\_effort =} \FunctionTok{n\_distinct}\NormalTok{(Sampling\_effort),}
         \CommentTok{\# get the maximum sampling effort for the studies}
         
         \AttributeTok{max\_sample\_effort =} \FunctionTok{max}\NormalTok{(Sampling\_effort)}
\NormalTok{  ) }\SpecialCharTok{\%\textgreater{}\%}
  
  \FunctionTok{ungroup}\NormalTok{() }\SpecialCharTok{\%\textgreater{}\%}
  
  \CommentTok{\# if the study has more than one sampling effort, correct the abundance}
  
  \CommentTok{\# so if there\textquotesingle{}s only one sampling effort, then create a \textquotesingle{}dummy sampling effort\textquotesingle{} of 1 so that we don\textquotesingle{}t change the abundances when we do the divisions. Otherwise, we give it the maximum sampling effort.}
  \FunctionTok{mutate}\NormalTok{(}\AttributeTok{DividingEffort =} \FunctionTok{ifelse}\NormalTok{(n\_sample\_effort }\SpecialCharTok{==} \DecValTok{1}\NormalTok{, }\DecValTok{1}\NormalTok{, max\_sample\_effort)) }\SpecialCharTok{\%\textgreater{}\%}
  
  \CommentTok{\# if the diversity metric isn\textquotesingle{}t sensitive to the effort, then we\textquotesingle{}ll change the value to 1 too (so we won\textquotesingle{}t end up changing the measurement), otherwise leave it as it is}
  \FunctionTok{mutate}\NormalTok{(}\AttributeTok{DividingEffort =} \FunctionTok{ifelse}\NormalTok{(Diversity\_metric\_is\_effort\_sensitive }\SpecialCharTok{==} \ConstantTok{FALSE}\NormalTok{, }\DecValTok{1}\NormalTok{, DividingEffort)) }\SpecialCharTok{\%\textgreater{}\%}
  
  \CommentTok{\# now let\textquotesingle{}s create the effort corrected measurement by dividing the abundances by the sampling efforts}
  \FunctionTok{mutate}\NormalTok{(}\AttributeTok{Corrected\_sampling\_effort =}\NormalTok{ Sampling\_effort }\SpecialCharTok{/}\NormalTok{ max\_sample\_effort,}
         \AttributeTok{Effort\_corrected\_measurement =}\NormalTok{ Measurement }\SpecialCharTok{*}\NormalTok{ Corrected\_sampling\_effort) }

\FunctionTok{summary}\NormalTok{(relevled\_diversity}\SpecialCharTok{$}\NormalTok{Corrected\_sampling\_effort)}
\end{Highlighting}
\end{Shaded}

\begin{verbatim}
##    Min. 1st Qu.  Median    Mean 3rd Qu.    Max. 
##   0.125   1.000   1.000   0.988   1.000   1.000
\end{verbatim}

We need to test if the \texttt{Corrected\_sampling\_effort} is working,
I know that LC1\_2015\_\_Su 1 has multiple sampling efforts during the
data inputting, let's check it to see if it works.

\begin{Shaded}
\begin{Highlighting}[]
\NormalTok{relevled\_diversity }\SpecialCharTok{\%\textgreater{}\%} 
  \CommentTok{\# filter out the test study}
  \FunctionTok{filter}\NormalTok{(SS }\SpecialCharTok{==} \StringTok{"LC1\_2015\_\_Su 1"}\NormalTok{) }\SpecialCharTok{\%\textgreater{}\%} 
  \CommentTok{\# select out the columns to check}
\NormalTok{  dplyr}\SpecialCharTok{::}\FunctionTok{select}\NormalTok{(Measurement, Effort\_corrected\_measurement, Sampling\_effort, max\_sample\_effort, Corrected\_sampling\_effort)}
\end{Highlighting}
\end{Shaded}

\begin{verbatim}
## # A tibble: 5,490 x 5
##    Measurement Effort_corrected_measurement Sampling_effort max_sample_effort
##          <dbl>                        <dbl>           <dbl>             <dbl>
##  1           0                            0               5                30
##  2           0                            0               5                30
##  3           0                            0               5                30
##  4           0                            0               5                30
##  5           0                            0               5                30
##  6           0                            0               5                30
##  7           0                            0               5                30
##  8           0                            0               5                30
##  9           0                            0               5                30
## 10           0                            0               5                30
## # ... with 5,480 more rows, and 1 more variable:
## #   Corrected_sampling_effort <dbl>
\end{verbatim}

It is working, now let's check studies that have same sampling effort,
here I test LC1\_2021\_\_Ge 1 since I know it has the same sampling
efforts during data inputting.

\begin{Shaded}
\begin{Highlighting}[]
\NormalTok{relevled\_diversity }\SpecialCharTok{\%\textgreater{}\%} 
  \FunctionTok{filter}\NormalTok{(SS }\SpecialCharTok{==} \StringTok{"LC1\_2021\_\_Ge 1"}\NormalTok{) }\SpecialCharTok{\%\textgreater{}\%} 
\NormalTok{  dplyr}\SpecialCharTok{::}\FunctionTok{select}\NormalTok{(Measurement, Effort\_corrected\_measurement)}
\end{Highlighting}
\end{Shaded}

\begin{verbatim}
## # A tibble: 350 x 2
##    Measurement Effort_corrected_measurement
##          <dbl>                        <dbl>
##  1           0                            0
##  2           0                            0
##  3           0                            0
##  4           0                            0
##  5           1                            1
##  6           0                            0
##  7           0                            0
##  8           0                            0
##  9           0                            0
## 10           1                            1
## # ... with 340 more rows
\end{verbatim}

\hypertarget{merge-sites}{%
\subsection{Merge sites}\label{merge-sites}}

Next we'll merge any sites that are within the same land-use type and
that have identical coordinates, start and end dates. This is to deal
with such situations, for instance, authors might put 10 pit traps
seperately as ten sampling points. Then, it looks like we have mores
sites but actually they have same land use, same use intensity and same
coordinates, which means they are just one site.

\begin{Shaded}
\begin{Highlighting}[]
\NormalTok{relevled\_diversity }\OtherTok{\textless{}{-}}\NormalTok{ relevled\_diversity }\SpecialCharTok{\%\textgreater{}\%}
  
  \CommentTok{\# group by aspects of the sites that should be identical if we need to merge the abundances}
  \CommentTok{\# I only want to merge abundances if they are within the same study and block}
  \CommentTok{\# as I\textquotesingle{}m assuming that even if the locations and sampling times are the same, if the blocks or studies are different, then there is some good reason for this.}
  \FunctionTok{group\_by}\NormalTok{(Source\_ID, Study\_number, Study\_name, Block,}
           \CommentTok{\#diversity metric type}
\NormalTok{           Diversity\_metric, Diversity\_metric\_type, Diversity\_metric\_unit,}
\NormalTok{           Diversity\_metric\_is\_effort\_sensitive,}
           
           \CommentTok{\#details of the sites}
\NormalTok{           Predominant\_habitat, Use\_intensity, Years\_since\_fragmentation\_or\_conversion,}
           
           \CommentTok{\#details of the sampling method}
\NormalTok{           Sampling\_method, Sampling\_effort\_unit,}
           
           \CommentTok{\#species identity}
\NormalTok{           Study\_common\_taxon, Rank\_of\_study\_common\_taxon,}
\NormalTok{           Taxon\_number, Taxon\_name\_entered,}
\NormalTok{           Indication, Parsed\_name,}
\NormalTok{           Best\_guess\_binomial, COL\_ID, Taxon, Name\_status,}
\NormalTok{           Rank, Kingdom, Phylum, Class, Order, Family, Genus, Species,}
\NormalTok{           Higher\_taxon,}
           
           \CommentTok{\#site location}
\NormalTok{           Longitude, Latitude,}
           
           \CommentTok{\#sampling time}
\NormalTok{           Sample\_start\_earliest, Sample\_end\_latest, Sample\_date\_resolution) }\SpecialCharTok{\%\textgreater{}\%}
  
  \CommentTok{\# if the diversity metric is occurrence:}
  \CommentTok{\#   if it is present at all, give it a 1, if it is always absent, give it a 0,}
  \CommentTok{\# otherwise (if the metric is either abundance or species richness):}
  \CommentTok{\#   calculate the weighted abundance/richness for each taxonomic group, weighted by sampling effort}
  
  \FunctionTok{mutate}\NormalTok{(}\AttributeTok{merged\_diversity =} 
           \FunctionTok{ifelse}\NormalTok{(Diversity\_metric\_type }\SpecialCharTok{==} \StringTok{"Occurrence"}\NormalTok{,}
                  \CommentTok{\# if any of the occurrence values are 1, \textasciigrave{}any\textasciigrave{} will return TRUE. If you sum a logical, TRUE becomes 1 and FALSE becomes 0}
                  \FunctionTok{sum}\NormalTok{(}\FunctionTok{any}\NormalTok{(Effort\_corrected\_measurement }\SpecialCharTok{\textgreater{}} \DecValTok{0}\NormalTok{)),}
                  
                  \CommentTok{\# note that since we\textquotesingle{}ve already corrected the sampling effort, this is essentially a mean rather than a weighted mean for abundance measurements. It\textquotesingle{}s a weighted mean for species richness though where sampling efforts vary.}
\NormalTok{                  stats}\SpecialCharTok{::}\FunctionTok{weighted.mean}\NormalTok{(}\AttributeTok{x =}\NormalTok{ Effort\_corrected\_measurement,}
                                       \AttributeTok{w =}\NormalTok{ Corrected\_sampling\_effort))}
\NormalTok{         )}

\CommentTok{\# pull out the grouping data (so we can double check how many records we\textquotesingle{}re merging for each)}
\NormalTok{group\_dat }\OtherTok{\textless{}{-}}\NormalTok{ relevled\_diversity }\SpecialCharTok{\%\textgreater{}\%} 
  \FunctionTok{group\_data}\NormalTok{() }\SpecialCharTok{\%\textgreater{}\%}
  \FunctionTok{mutate}\NormalTok{(}\AttributeTok{nvals\_merged =} \FunctionTok{lengths}\NormalTok{(.rows),}
         \AttributeTok{merge\_ID =} \FunctionTok{row\_number}\NormalTok{())}

\CommentTok{\# ungroup the relevled\_diversity data for future use}
\NormalTok{relevled\_diversity }\OtherTok{\textless{}{-}} \FunctionTok{ungroup}\NormalTok{(relevled\_diversity)}

\CommentTok{\# create a dataset where we can extract just the merged data if we want to}
\NormalTok{diversity\_merged }\OtherTok{\textless{}{-}}\NormalTok{ relevled\_diversity }\SpecialCharTok{\%\textgreater{}\%}
  \FunctionTok{left\_join}\NormalTok{(group\_dat)}
\end{Highlighting}
\end{Shaded}

\begin{verbatim}
## Joining, by = c("Source_ID", "Study_number", "Study_name", "Diversity_metric", "Diversity_metric_unit", "Diversity_metric_type", "Diversity_metric_is_effort_sensitive", "Sampling_method", "Sampling_effort_unit", "Block", "Sample_start_earliest", "Sample_end_latest", "Sample_date_resolution", "Predominant_habitat", "Use_intensity", "Years_since_fragmentation_or_conversion", "Longitude", "Latitude", "Taxon_number", "Taxon_name_entered", "Indication", "Parsed_name", "COL_ID", "Taxon", "Name_status", "Rank", "Kingdom", "Phylum", "Class", "Order", "Family", "Genus", "Species", "Higher_taxon", "Study_common_taxon", "Rank_of_study_common_taxon", "Best_guess_binomial")
\end{verbatim}

\begin{Shaded}
\begin{Highlighting}[]
\CommentTok{\# check that the merging has worked (row numbers should be equal right now)}
\FunctionTok{nrow}\NormalTok{(relevled\_diversity) }\SpecialCharTok{==} \FunctionTok{nrow}\NormalTok{(diversity\_merged)}
\end{Highlighting}
\end{Shaded}

\begin{verbatim}
## [1] TRUE
\end{verbatim}

Now let's test that the sites have been merged correctly.

\begin{Shaded}
\begin{Highlighting}[]
\NormalTok{test\_data }\OtherTok{\textless{}{-}}\NormalTok{ diversity\_merged }\SpecialCharTok{\%\textgreater{}\%}
  \FunctionTok{filter}\NormalTok{(nvals\_merged }\SpecialCharTok{\textgreater{}} \DecValTok{1}\NormalTok{) }\SpecialCharTok{\%\textgreater{}\%}
  \FunctionTok{distinct}\NormalTok{(merge\_ID, }\AttributeTok{.keep\_all =} \ConstantTok{TRUE}\NormalTok{)}

\NormalTok{test1 }\OtherTok{\textless{}{-}}\NormalTok{ diversity\_merged }\SpecialCharTok{\%\textgreater{}\%}
  \FunctionTok{filter}\NormalTok{(merge\_ID }\SpecialCharTok{==}\NormalTok{ test\_data}\SpecialCharTok{$}\NormalTok{merge\_ID[}\DecValTok{1}\NormalTok{]) }\SpecialCharTok{\%\textgreater{}\%}
\NormalTok{  dplyr}\SpecialCharTok{::}\FunctionTok{select}\NormalTok{(SS, SSB,}
\NormalTok{         Diversity\_metric, Diversity\_metric\_type, Diversity\_metric\_unit,}
\NormalTok{         Predominant\_habitat, Use\_intensity, Years\_since\_fragmentation\_or\_conversion,}
\NormalTok{         Sampling\_method, Sampling\_effort\_unit,}
\NormalTok{         Study\_common\_taxon, Rank\_of\_study\_common\_taxon,}
\NormalTok{         Taxon\_name\_entered,}
\NormalTok{         Best\_guess\_binomial,}
\NormalTok{         Longitude, Latitude,}
\NormalTok{         Sample\_start\_earliest, Sample\_end\_latest, Sample\_date\_resolution,}
\NormalTok{         Effort\_corrected\_measurement,}
\NormalTok{         Corrected\_sampling\_effort,}
\NormalTok{         merged\_diversity,}
\NormalTok{         .rows,}
\NormalTok{         nvals\_merged,}
\NormalTok{         merge\_ID}
\NormalTok{         )}

\NormalTok{test1}
\end{Highlighting}
\end{Shaded}

\begin{verbatim}
## # A tibble: 10 x 25
##    SS                  SSB    Diversity_metric Diversity_metri~ Diversity_metri~
##    <fct>               <fct>  <fct>            <fct>            <fct>           
##  1 AD2_2013__Schmidt 1 AD2_2~ occurrence       Occurrence       presence/absence
##  2 AD2_2013__Schmidt 1 AD2_2~ occurrence       Occurrence       presence/absence
##  3 AD2_2013__Schmidt 1 AD2_2~ occurrence       Occurrence       presence/absence
##  4 AD2_2013__Schmidt 1 AD2_2~ occurrence       Occurrence       presence/absence
##  5 AD2_2013__Schmidt 1 AD2_2~ occurrence       Occurrence       presence/absence
##  6 AD2_2013__Schmidt 1 AD2_2~ occurrence       Occurrence       presence/absence
##  7 AD2_2013__Schmidt 1 AD2_2~ occurrence       Occurrence       presence/absence
##  8 AD2_2013__Schmidt 1 AD2_2~ occurrence       Occurrence       presence/absence
##  9 AD2_2013__Schmidt 1 AD2_2~ occurrence       Occurrence       presence/absence
## 10 AD2_2013__Schmidt 1 AD2_2~ occurrence       Occurrence       presence/absence
## # ... with 20 more variables: Predominant_habitat <fct>, Use_intensity <fct>,
## #   Years_since_fragmentation_or_conversion <dbl>, Sampling_method <fct>,
## #   Sampling_effort_unit <fct>, Study_common_taxon <fct>,
## #   Rank_of_study_common_taxon <fct>, Taxon_name_entered <fct>,
## #   Best_guess_binomial <fct>, Longitude <dbl>, Latitude <dbl>,
## #   Sample_start_earliest <date>, Sample_end_latest <date>,
## #   Sample_date_resolution <fct>, Effort_corrected_measurement <dbl>, ...
\end{verbatim}

\begin{Shaded}
\begin{Highlighting}[]
\FunctionTok{paste}\NormalTok{(test\_data}\SpecialCharTok{$}\NormalTok{SS[}\DecValTok{8000}\NormalTok{])}
\end{Highlighting}
\end{Shaded}

\begin{verbatim}
## [1] "AD2_2017__Novais 1"
\end{verbatim}

\begin{Shaded}
\begin{Highlighting}[]
\NormalTok{test2 }\OtherTok{\textless{}{-}}\NormalTok{ diversity\_merged }\SpecialCharTok{\%\textgreater{}\%}
  \FunctionTok{filter}\NormalTok{(merge\_ID }\SpecialCharTok{==}\NormalTok{ test\_data}\SpecialCharTok{$}\NormalTok{merge\_ID[}\DecValTok{8000}\NormalTok{]) }\SpecialCharTok{\%\textgreater{}\%}
\NormalTok{  dplyr}\SpecialCharTok{::}\FunctionTok{select}\NormalTok{(SS, SSB,}
\NormalTok{         Diversity\_metric, Diversity\_metric\_type, Diversity\_metric\_unit,}
\NormalTok{         Predominant\_habitat, Use\_intensity, Years\_since\_fragmentation\_or\_conversion,}
\NormalTok{         Sampling\_method, Sampling\_effort\_unit,}
\NormalTok{         Study\_common\_taxon, Rank\_of\_study\_common\_taxon,}
\NormalTok{         Taxon\_name\_entered,}
\NormalTok{         Best\_guess\_binomial,}
\NormalTok{         Longitude, Latitude,}
\NormalTok{         Sample\_start\_earliest, Sample\_end\_latest, Sample\_date\_resolution,}
\NormalTok{         Effort\_corrected\_measurement,}
\NormalTok{         Corrected\_sampling\_effort,}
\NormalTok{         merged\_diversity,}
\NormalTok{         .rows,}
\NormalTok{         nvals\_merged,}
\NormalTok{         merge\_ID}
\NormalTok{         )}

\NormalTok{test2}
\end{Highlighting}
\end{Shaded}

\begin{verbatim}
## # A tibble: 6 x 25
##   SS                 SSB     Diversity_metric Diversity_metric~ Diversity_metri~
##   <fct>              <fct>   <fct>            <fct>             <fct>           
## 1 AD2_2017__Novais 1 "AD2_2~ abundance        Abundance         individuals     
## 2 AD2_2017__Novais 1 "AD2_2~ abundance        Abundance         individuals     
## 3 AD2_2017__Novais 1 "AD2_2~ abundance        Abundance         individuals     
## 4 AD2_2017__Novais 1 "AD2_2~ abundance        Abundance         individuals     
## 5 AD2_2017__Novais 1 "AD2_2~ abundance        Abundance         individuals     
## 6 AD2_2017__Novais 1 "AD2_2~ abundance        Abundance         individuals     
## # ... with 20 more variables: Predominant_habitat <fct>, Use_intensity <fct>,
## #   Years_since_fragmentation_or_conversion <dbl>, Sampling_method <fct>,
## #   Sampling_effort_unit <fct>, Study_common_taxon <fct>,
## #   Rank_of_study_common_taxon <fct>, Taxon_name_entered <fct>,
## #   Best_guess_binomial <fct>, Longitude <dbl>, Latitude <dbl>,
## #   Sample_start_earliest <date>, Sample_end_latest <date>,
## #   Sample_date_resolution <fct>, Effort_corrected_measurement <dbl>, ...
\end{verbatim}

Now let's check some studies that didn't need merging.

\begin{Shaded}
\begin{Highlighting}[]
\NormalTok{test\_data }\OtherTok{\textless{}{-}}\NormalTok{ diversity\_merged }\SpecialCharTok{\%\textgreater{}\%}
  \FunctionTok{filter}\NormalTok{(nvals\_merged }\SpecialCharTok{==} \DecValTok{1}\NormalTok{) }\SpecialCharTok{\%\textgreater{}\%}
  \FunctionTok{distinct}\NormalTok{(merge\_ID, }\AttributeTok{.keep\_all =} \ConstantTok{TRUE}\NormalTok{)}

\FunctionTok{paste}\NormalTok{(test\_data}\SpecialCharTok{$}\NormalTok{SS[}\DecValTok{1}\NormalTok{])}
\end{Highlighting}
\end{Shaded}

\begin{verbatim}
## [1] "AD1_2002__Tonhasca 1"
\end{verbatim}

\begin{Shaded}
\begin{Highlighting}[]
\NormalTok{test3 }\OtherTok{\textless{}{-}}\NormalTok{ diversity\_merged }\SpecialCharTok{\%\textgreater{}\%}
  \FunctionTok{filter}\NormalTok{(merge\_ID }\SpecialCharTok{==}\NormalTok{ test\_data}\SpecialCharTok{$}\NormalTok{merge\_ID[}\DecValTok{1}\NormalTok{]) }\SpecialCharTok{\%\textgreater{}\%}
\NormalTok{  dplyr}\SpecialCharTok{::}\FunctionTok{select}\NormalTok{(SS, SSB,}
\NormalTok{         Diversity\_metric, Diversity\_metric\_type, Diversity\_metric\_unit,}
\NormalTok{         Predominant\_habitat, Use\_intensity, Years\_since\_fragmentation\_or\_conversion,}
\NormalTok{         Sampling\_method, Sampling\_effort\_unit,}
\NormalTok{         Study\_common\_taxon, Rank\_of\_study\_common\_taxon,}
\NormalTok{         Taxon\_name\_entered,}
\NormalTok{         Best\_guess\_binomial,}
\NormalTok{         Longitude, Latitude,}
\NormalTok{         Sample\_start\_earliest, Sample\_end\_latest, Sample\_date\_resolution,}
\NormalTok{         Effort\_corrected\_measurement,}
\NormalTok{         Corrected\_sampling\_effort,}
\NormalTok{         merged\_diversity,}
\NormalTok{         .rows,}
\NormalTok{         nvals\_merged,}
\NormalTok{         merge\_ID}
\NormalTok{         )}

\NormalTok{test3}
\end{Highlighting}
\end{Shaded}

\begin{verbatim}
## # A tibble: 1 x 25
##   SS                   SSB    Diversity_metric Diversity_metri~ Diversity_metri~
##   <fct>                <fct>  <fct>            <fct>            <fct>           
## 1 AD1_2002__Tonhasca 1 "AD1_~ abundance        Abundance        individuals     
## # ... with 20 more variables: Predominant_habitat <fct>, Use_intensity <fct>,
## #   Years_since_fragmentation_or_conversion <dbl>, Sampling_method <fct>,
## #   Sampling_effort_unit <fct>, Study_common_taxon <fct>,
## #   Rank_of_study_common_taxon <fct>, Taxon_name_entered <fct>,
## #   Best_guess_binomial <fct>, Longitude <dbl>, Latitude <dbl>,
## #   Sample_start_earliest <date>, Sample_end_latest <date>,
## #   Sample_date_resolution <fct>, Effort_corrected_measurement <dbl>, ...
\end{verbatim}

Everything is working as planed, let's go for abundance and richness
data.

\hypertarget{calculate-abundance-and-richness-data}{%
\subsection{Calculate abundance and richness
data}\label{calculate-abundance-and-richness-data}}

\begin{Shaded}
\begin{Highlighting}[]
\NormalTok{sites }\OtherTok{\textless{}{-}}\NormalTok{ diversity\_merged }\SpecialCharTok{\%\textgreater{}\%}
  
  \CommentTok{\# pull out only the merged diversity data}
  \FunctionTok{distinct}\NormalTok{(merge\_ID, }\AttributeTok{.keep\_all =} \ConstantTok{TRUE}\NormalTok{) }\SpecialCharTok{\%\textgreater{}\%}
  
  \CommentTok{\# re{-}make SSB and SSBS values since we\textquotesingle{}ve now dropped a bunch of values}
  \FunctionTok{mutate}\NormalTok{(}\AttributeTok{SS =} \FunctionTok{paste}\NormalTok{(Source\_ID, Study\_number),}
         \AttributeTok{SSB =} \FunctionTok{paste}\NormalTok{(SS, Block),}
         \AttributeTok{SSBS =} \FunctionTok{paste}\NormalTok{(SSB, Site\_number)) }\SpecialCharTok{\%\textgreater{}\%}
  
  \CommentTok{\# group by SSBS (each unique value corresponds to a unique site)}
  \FunctionTok{group\_by}\NormalTok{(SSBS) }\SpecialCharTok{\%\textgreater{}\%}
  
  \CommentTok{\# now add up all the abundance measurements within each site}
  \FunctionTok{mutate}\NormalTok{(}\AttributeTok{TotalAbundance =} \FunctionTok{ifelse}\NormalTok{(Diversity\_metric\_type }\SpecialCharTok{==} \StringTok{"Abundance"}\NormalTok{,}
                                 \FunctionTok{sum}\NormalTok{(merged\_diversity),}
                                 \CommentTok{\# if the diversity metric type isn\textquotesingle{}t Abundance, then leave the TotalAbundance measurement as NA}
                                 \ConstantTok{NA}\NormalTok{),}
         
         \AttributeTok{SpeciesRichness =} \FunctionTok{ifelse}\NormalTok{(Diversity\_metric\_type }\SpecialCharTok{==} \StringTok{"Species richness"}\NormalTok{,}
\NormalTok{                                  merged\_diversity,}
                                  \CommentTok{\# for abundance and occurrence measurements, count the number of unique species names that are present at the site }
                                  \FunctionTok{n\_distinct}\NormalTok{(Best\_guess\_binomial[merged\_diversity }\SpecialCharTok{\textgreater{}} \DecValTok{0}\NormalTok{]))) }\SpecialCharTok{\%\textgreater{}\%}
  
  \CommentTok{\# ungroup}
  \FunctionTok{ungroup}\NormalTok{() }\SpecialCharTok{\%\textgreater{}\%}
  
  \CommentTok{\# pull out unique sites}
  \FunctionTok{distinct}\NormalTok{(SSBS, }\AttributeTok{.keep\_all =} \ConstantTok{TRUE}\NormalTok{) }\SpecialCharTok{\%\textgreater{}\%}
  
  \CommentTok{\# now group by Study ID}
  \FunctionTok{group\_by}\NormalTok{(SS) }\SpecialCharTok{\%\textgreater{}\%}
  
  \CommentTok{\# pull out the maximum abundance for each study}
  \FunctionTok{mutate}\NormalTok{(}\AttributeTok{MaxAbundance =} \FunctionTok{max}\NormalTok{(TotalAbundance)) }\SpecialCharTok{\%\textgreater{}\%}
  
  \CommentTok{\# ungroup}
  \FunctionTok{ungroup}\NormalTok{() }\SpecialCharTok{\%\textgreater{}\%}
  
  \CommentTok{\# now rescale total abundance, so that within each study, abundance varies from 0 to 1.}
  \FunctionTok{mutate}\NormalTok{(}\AttributeTok{RescaledAbundance =}\NormalTok{ TotalAbundance}\SpecialCharTok{/}\NormalTok{MaxAbundance)}
\end{Highlighting}
\end{Shaded}

\hypertarget{start-to-model}{%
\subsection{Start to model}\label{start-to-model}}

\hypertarget{collinearity}{%
\subsection{collinearity}\label{collinearity}}

Check if there is collinearity for the variables we are going to use.

\begin{Shaded}
\begin{Highlighting}[]
\FunctionTok{source}\NormalTok{(}\StringTok{"https://highstat.com/Books/Book2/HighstatLibV10.R"}\NormalTok{)}

\FunctionTok{corvif}\NormalTok{(sites[ , }\FunctionTok{c}\NormalTok{(}\StringTok{"LandUse"}\NormalTok{, }\StringTok{"UseIntensity"}\NormalTok{, }\StringTok{"Country"}\NormalTok{)])}
\end{Highlighting}
\end{Shaded}

\begin{verbatim}
## 
## 
## Variance inflation factors
## 
##                  GVIF Df GVIF^(1/2Df)
## LandUse      3.186850  5     1.122887
## UseIntensity 1.970537  2     1.184803
## Country      1.664981  1     1.290341
\end{verbatim}

From the GVIFs, LandUse, Useintensity and Country all have values
smaller than 3 which indicates the collinearity check is ok.

\hypertarget{complete-cases}{%
\subsection{complete cases}\label{complete-cases}}

Before the modelling, we need to drop the row that has NA in our
explanatory variables which is RescaledAbundance, LandUse, UseIntensity
and Country.

\begin{Shaded}
\begin{Highlighting}[]
\NormalTok{model\_data\_ab }\OtherTok{\textless{}{-}} \FunctionTok{drop\_na}\NormalTok{(sites, }
\NormalTok{                         RescaledAbundance, LandUse,}
\NormalTok{                         UseIntensity, Country)}

\NormalTok{model\_data\_sr }\OtherTok{\textless{}{-}} \FunctionTok{drop\_na}\NormalTok{(sites, }
\NormalTok{                         SpeciesRichness, LandUse,}
\NormalTok{                         UseIntensity, Country)}
\end{Highlighting}
\end{Shaded}

\hypertarget{modeling}{%
\subsection{Modeling}\label{modeling}}

We will analyze the abundance data first. lmer model is used for
abundance data, we should check the distribution of residuals first.

\begin{Shaded}
\begin{Highlighting}[]
\NormalTok{mab }\OtherTok{\textless{}{-}} \FunctionTok{lmer}\NormalTok{(RescaledAbundance }\SpecialCharTok{\textasciitilde{}}\NormalTok{ LandUse }\SpecialCharTok{+}\NormalTok{ UseIntensity }\SpecialCharTok{+}\NormalTok{ Country }\SpecialCharTok{+}\NormalTok{ (}\DecValTok{1}\SpecialCharTok{|}\NormalTok{SS), }\AttributeTok{data =}\NormalTok{ model\_data\_ab)}
\NormalTok{simulationOutput }\OtherTok{\textless{}{-}}\FunctionTok{simulateResiduals}\NormalTok{(}\AttributeTok{fittedModel =}\NormalTok{ mab, }\AttributeTok{plot =}\NormalTok{ F)}
\FunctionTok{plot}\NormalTok{(simulationOutput)}\CommentTok{\# check the residuals plot}
\end{Highlighting}
\end{Shaded}

\includegraphics{Analysis-for-insecta-data_files/figure-latex/unnamed-chunk-22-1.pdf}

Based on the above plots, log transformation can help. To avoid log(0),
0.01 is added to `RescaledAbundance'.

\begin{Shaded}
\begin{Highlighting}[]
\NormalTok{model\_data\_ab }\OtherTok{\textless{}{-}} \FunctionTok{mutate}\NormalTok{(model\_data\_ab, }
                        \AttributeTok{logAbundance =} \FunctionTok{log}\NormalTok{(}\FloatTok{0.01} \SpecialCharTok{+}\NormalTok{ RescaledAbundance))}
\NormalTok{mab }\OtherTok{\textless{}{-}} \FunctionTok{lmer}\NormalTok{(logAbundance }\SpecialCharTok{\textasciitilde{}}\NormalTok{ LandUse }\SpecialCharTok{+}\NormalTok{ UseIntensity }\SpecialCharTok{+}\NormalTok{ Country }\SpecialCharTok{+}\NormalTok{ (}\DecValTok{1}\SpecialCharTok{|}\NormalTok{SS), }\AttributeTok{data =}\NormalTok{ model\_data\_ab)}
\NormalTok{simulationOutput }\OtherTok{\textless{}{-}}\FunctionTok{simulateResiduals}\NormalTok{(}\AttributeTok{fittedModel =}\NormalTok{ mab, }\AttributeTok{plot =}\NormalTok{ F)}
\FunctionTok{plot}\NormalTok{(simulationOutput)}\CommentTok{\# check the residuals plot}
\end{Highlighting}
\end{Shaded}

\includegraphics{Analysis-for-insecta-data_files/figure-latex/unnamed-chunk-23-1.pdf}

This is ok to work now. Next we build a model for richness data. Since
it is count data, we will use glmer and poisson errors. For posisson
error structure, we need to check for overdispersion first.

\begin{Shaded}
\begin{Highlighting}[]
\NormalTok{msr }\OtherTok{\textless{}{-}} \FunctionTok{glmer}\NormalTok{(SpeciesRichness }\SpecialCharTok{\textasciitilde{}}\NormalTok{ LandUse }\SpecialCharTok{+}\NormalTok{ UseIntensity }\SpecialCharTok{+}\NormalTok{ Country }\SpecialCharTok{+}\NormalTok{ (}\DecValTok{1}\SpecialCharTok{|}\NormalTok{SS) }\SpecialCharTok{+}\NormalTok{ (}\DecValTok{1}\SpecialCharTok{|}\NormalTok{SSB), }\AttributeTok{data =}\NormalTok{ model\_data\_sr,}\AttributeTok{family =} \StringTok{"poisson"}\NormalTok{)}
\CommentTok{\# check overdispersion}
\CommentTok{\# load in an overdispersion function from here: https://bbolker.github.io/mixedmodels{-}misc/glmmFAQ.html\#testing{-}for{-}overdispersioncomputing{-}overdispersion{-}factor}
\NormalTok{overdisp\_fun }\OtherTok{\textless{}{-}} \ControlFlowTok{function}\NormalTok{(model) \{}
\NormalTok{    rdf }\OtherTok{\textless{}{-}} \FunctionTok{df.residual}\NormalTok{(model)}
\NormalTok{    rp }\OtherTok{\textless{}{-}} \FunctionTok{residuals}\NormalTok{(model,}\AttributeTok{type=}\StringTok{"pearson"}\NormalTok{)}
\NormalTok{    Pearson.chisq }\OtherTok{\textless{}{-}} \FunctionTok{sum}\NormalTok{(rp}\SpecialCharTok{\^{}}\DecValTok{2}\NormalTok{)}
\NormalTok{    prat }\OtherTok{\textless{}{-}}\NormalTok{ Pearson.chisq}\SpecialCharTok{/}\NormalTok{rdf}
\NormalTok{    pval }\OtherTok{\textless{}{-}} \FunctionTok{pchisq}\NormalTok{(Pearson.chisq, }\AttributeTok{df=}\NormalTok{rdf, }\AttributeTok{lower.tail=}\ConstantTok{FALSE}\NormalTok{)}
    \FunctionTok{c}\NormalTok{(}\AttributeTok{chisq=}\NormalTok{Pearson.chisq,}\AttributeTok{ratio=}\NormalTok{prat,}\AttributeTok{rdf=}\NormalTok{rdf,}\AttributeTok{p=}\NormalTok{pval)}
\NormalTok{\}}

\CommentTok{\# test the species richness model for overdispersion}
\FunctionTok{overdisp\_fun}\NormalTok{(msr)}
\end{Highlighting}
\end{Shaded}

\begin{verbatim}
##        chisq        ratio          rdf            p 
## 1501.3467993    0.9982359 1504.0000000    0.5144573
\end{verbatim}

P value is 0.514 which indicates the model is fine. My study also want
to compare a model in which Country interacts with the other fixed
effects (*), vs a model in which Country is only an additive effect (+).

\begin{Shaded}
\begin{Highlighting}[]
\NormalTok{mab2 }\OtherTok{\textless{}{-}} \FunctionTok{lmer}\NormalTok{(logAbundance }\SpecialCharTok{\textasciitilde{}}\NormalTok{ (LandUse }\SpecialCharTok{+}\NormalTok{ UseIntensity) }\SpecialCharTok{*}\NormalTok{ Country }\SpecialCharTok{+}\NormalTok{ (}\DecValTok{1}\SpecialCharTok{|}\NormalTok{SS), }\AttributeTok{data =}\NormalTok{ model\_data\_ab)}

\NormalTok{msr2 }\OtherTok{\textless{}{-}} \FunctionTok{glmer}\NormalTok{(SpeciesRichness }\SpecialCharTok{\textasciitilde{}}\NormalTok{ (LandUse }\SpecialCharTok{+}\NormalTok{ UseIntensity) }\SpecialCharTok{*}\NormalTok{ Country }\SpecialCharTok{+}\NormalTok{ (}\DecValTok{1}\SpecialCharTok{|}\NormalTok{SS) }\SpecialCharTok{+}\NormalTok{ (}\DecValTok{1}\SpecialCharTok{|}\NormalTok{SSB), }\AttributeTok{data =}\NormalTok{ model\_data\_sr, }\AttributeTok{family =} \StringTok{"poisson"}\NormalTok{, }\AttributeTok{control =} \FunctionTok{glmerControl}\NormalTok{(}\AttributeTok{optimizer =} \StringTok{"bobyqa"}\NormalTok{, }\AttributeTok{optCtrl =} \FunctionTok{list}\NormalTok{(}\AttributeTok{maxfun =} \DecValTok{20000}\NormalTok{))) }\CommentTok{\# the following code: control = glmerControl(optimizer = "bobyqa", optCtrl = list(maxfun = 20000), is added since this model cannot converge}


\FunctionTok{anova}\NormalTok{(mab, mab2, }\AttributeTok{refit =} \ConstantTok{FALSE}\NormalTok{) }\CommentTok{\# refit = FALSE is added due to the warning message: refitting model(s) with ML (instead of REML)}
\end{Highlighting}
\end{Shaded}

\begin{verbatim}
## Data: model_data_ab
## Models:
## mab: logAbundance ~ LandUse + UseIntensity + Country + (1 | SS)
## mab2: logAbundance ~ (LandUse + UseIntensity) * Country + (1 | SS)
##      npar    AIC    BIC  logLik deviance  Chisq Df Pr(>Chisq)
## mab    11 3526.5 3582.8 -1752.2   3504.5                     
## mab2   18 3529.6 3621.8 -1746.8   3493.6 10.846  7     0.1455
\end{verbatim}

\begin{Shaded}
\begin{Highlighting}[]
\FunctionTok{anova}\NormalTok{(msr, msr2)}
\end{Highlighting}
\end{Shaded}

\begin{verbatim}
## Data: model_data_sr
## Models:
## msr: SpeciesRichness ~ LandUse + UseIntensity + Country + (1 | SS) + (1 | SSB)
## msr2: SpeciesRichness ~ (LandUse + UseIntensity) * Country + (1 | SS) + (1 | SSB)
##      npar    AIC    BIC  logLik deviance  Chisq Df Pr(>Chisq)    
## msr    11 7533.7 7592.2 -3755.8   7511.7                         
## msr2   18 7510.9 7606.7 -3737.5   7474.9 36.739  7  5.257e-06 ***
## ---
## Signif. codes:  0 '***' 0.001 '**' 0.01 '*' 0.05 '.' 0.1 ' ' 1
\end{verbatim}

\hypertarget{reporting-the-results}{%
\subsection{Reporting the results}\label{reporting-the-results}}

For abundance data,

\begin{Shaded}
\begin{Highlighting}[]
\NormalTok{sjPlot}\SpecialCharTok{::}\FunctionTok{tab\_model}\NormalTok{(mab2, }
                  \AttributeTok{show.re.var=} \ConstantTok{TRUE}\NormalTok{)}
\end{Highlighting}
\end{Shaded}

~

logAbundance

Predictors

Estimates

CI

p

(Intercept)

-1.43

-1.66~--~-1.19

\textless0.001

LandUse {[}Cropland{]}

0.74

0.48~--~0.99

\textless0.001

LandUse {[}Pasture{]}

0.12

-0.13~--~0.37

0.361

LandUse {[}Plantationforest{]}

0.37

-0.04~--~0.79

0.080

LandUse {[}Secondaryvegetation{]}

0.46

0.20~--~0.72

0.001

LandUse {[}Urban{]}

0.15

-0.69~--~0.99

0.725

UseIntensity {[}Intenseuse{]}

-0.59

-1.04~--~-0.15

0.008

UseIntensity {[}Light use{]}

-0.11

-0.32~--~0.10

0.290

Country {[}China{]}

0.26

-0.88~--~1.39

0.658

LandUse {[}Cropland{]} *Country {[}China{]}

0.35

-0.90~--~1.60

0.585

LandUse {[}Pasture{]} *Country {[}China{]}

0.73

-0.53~--~1.99

0.259

LandUse {[}Plantationforest{]} * Country {[}China{]}

0.67

-0.63~--~1.97

0.312

LandUse {[}Secondaryvegetation{]} * Country{[}China{]}

0.25

-0.92~--~1.42

0.679

LandUse {[}Urban{]} * Country{[}China{]}

-0.02

-1.56~--~1.51

0.976

UseIntensity {[}Intenseuse{]} * Country {[}China{]}

0.08

-0.56~--~0.73

0.799

UseIntensity {[}Light use{]}* Country {[}China{]}

-0.56

-0.96~--~-0.16

0.007

Random Effects

σ2

0.87

τ00 SS

0.56

ICC

0.39

N SS

85

Observations

1236

Marginal R2 / Conditional R2

0.077 / 0.440

For richness data,

\begin{Shaded}
\begin{Highlighting}[]
\NormalTok{sjPlot}\SpecialCharTok{::}\FunctionTok{tab\_model}\NormalTok{(msr2, }
                  \AttributeTok{show.re.var=} \ConstantTok{TRUE}\NormalTok{)}
\end{Highlighting}
\end{Shaded}

~

SpeciesRichness

Predictors

Incidence Rate Ratios

CI

p

(Intercept)

7.02

5.39~--~9.13

\textless0.001

LandUse {[}Cropland{]}

1.12

1.03~--~1.23

0.011

LandUse {[}Pasture{]}

0.78

0.71~--~0.85

\textless0.001

LandUse {[}Plantationforest{]}

0.90

0.80~--~1.00

0.045

LandUse {[}Secondaryvegetation{]}

1.13

1.04~--~1.22

0.004

LandUse {[}Urban{]}

1.14

0.90~--~1.43

0.275

UseIntensity {[}Intenseuse{]}

0.85

0.73~--~0.98

0.023

UseIntensity {[}Light use{]}

1.13

1.05~--~1.21

0.001

Country {[}China{]}

0.61

0.33~--~1.15

0.129

LandUse {[}Cropland{]} *Country {[}China{]}

1.13

0.74~--~1.73

0.576

LandUse {[}Pasture{]} *Country {[}China{]}

1.21

0.75~--~1.97

0.438

LandUse {[}Plantationforest{]} * Country {[}China{]}

1.55

1.00~--~2.40

0.052

LandUse {[}Secondaryvegetation{]} * Country{[}China{]}

0.74

0.54~--~1.03

0.075

LandUse {[}Urban{]} * Country{[}China{]}

0.55

0.23~--~1.32

0.181

UseIntensity {[}Intenseuse{]} * Country {[}China{]}

0.77

0.58~--~1.01

0.062

UseIntensity {[}Light use{]}* Country {[}China{]}

0.63

0.52~--~0.77

\textless0.001

Random Effects

σ2

0.15

τ00 SSB

0.03

τ00 SS

1.31

ICC

0.90

N SS

103

N SSB

302

Observations

1515

Marginal R2 / Conditional R2

0.075 / 0.906

\begin{Shaded}
\begin{Highlighting}[]
\CommentTok{\# gather the effects and confidence intervals using simulation}
\NormalTok{sim\_eff\_ab }\OtherTok{\textless{}{-}} \FunctionTok{FEsim}\NormalTok{(mab2, }\DecValTok{1000}\NormalTok{)}

\NormalTok{sim\_eff\_ab }\SpecialCharTok{\%\textgreater{}\%}
  \CommentTok{\# drop the first row of the dataframe (the intercept value), since we\textquotesingle{}re plotting the effects \_compared\_ to this value}
  \FunctionTok{slice}\NormalTok{(}\SpecialCharTok{{-}}\FunctionTok{c}\NormalTok{(}\DecValTok{1}\NormalTok{)) }\SpecialCharTok{\%\textgreater{}\%}
  \FunctionTok{ggplot}\NormalTok{() }\SpecialCharTok{+}
  \CommentTok{\# plot the coefficients plus/minus 95\% confidence intervals (1.96 x standard deviation)}
  \FunctionTok{aes}\NormalTok{(}\AttributeTok{x =}\NormalTok{ term, }\AttributeTok{ymin =}\NormalTok{ median }\SpecialCharTok{{-}} \FloatTok{1.96} \SpecialCharTok{*}\NormalTok{ sd,}
      \AttributeTok{ymax =}\NormalTok{ median }\SpecialCharTok{+} \FloatTok{1.96} \SpecialCharTok{*}\NormalTok{ sd, }\AttributeTok{y =}\NormalTok{ median) }\SpecialCharTok{+}
  \FunctionTok{geom\_pointrange}\NormalTok{() }\SpecialCharTok{+}
  \CommentTok{\# add the line for the Intercept (Minimal Primary vegetation) at 0 (everything is compared to this)}
  \FunctionTok{geom\_hline}\NormalTok{(}\AttributeTok{yintercept =} \DecValTok{0}\NormalTok{, }\AttributeTok{size =} \FunctionTok{I}\NormalTok{(}\FloatTok{1.1}\NormalTok{), }\AttributeTok{color =} \FunctionTok{I}\NormalTok{(}\StringTok{"red"}\NormalTok{)) }\SpecialCharTok{+}
  \FunctionTok{coord\_flip}\NormalTok{() }\SpecialCharTok{+}
  \FunctionTok{theme\_bw}\NormalTok{() }\SpecialCharTok{+} \FunctionTok{labs}\NormalTok{(}\AttributeTok{x =} \StringTok{"Median Effect Estimate"}\NormalTok{, }\AttributeTok{y =} \StringTok{"log(Rescaled Abundance + 0.01)"}\NormalTok{)}
\end{Highlighting}
\end{Shaded}

\includegraphics{Analysis-for-insecta-data_files/figure-latex/unnamed-chunk-28-1.pdf}

Same plot for richness data

\begin{Shaded}
\begin{Highlighting}[]
\CommentTok{\# gather the effects and confidence intervals using simulation}
\NormalTok{sim\_eff\_sr }\OtherTok{\textless{}{-}} \FunctionTok{FEsim}\NormalTok{(msr2, }\DecValTok{1000}\NormalTok{)}

\NormalTok{sim\_eff\_sr }\SpecialCharTok{\%\textgreater{}\%}
  \CommentTok{\# drop the first row of the dataframe (the intercept value), since we\textquotesingle{}re plotting the effecs \_compared\_ to this value}
  \FunctionTok{slice}\NormalTok{(}\SpecialCharTok{{-}}\FunctionTok{c}\NormalTok{(}\DecValTok{1}\NormalTok{)) }\SpecialCharTok{\%\textgreater{}\%}
  \FunctionTok{ggplot}\NormalTok{() }\SpecialCharTok{+}
  \CommentTok{\# plot the coefficiets plus/minus 95\% confidence intervals (1.96 x standard deviation)}
  \FunctionTok{aes}\NormalTok{(}\AttributeTok{x =}\NormalTok{ term, }\AttributeTok{ymin =}\NormalTok{ median }\SpecialCharTok{{-}} \FloatTok{1.96} \SpecialCharTok{*}\NormalTok{ sd,}
      \AttributeTok{ymax =}\NormalTok{ median }\SpecialCharTok{+} \FloatTok{1.96} \SpecialCharTok{*}\NormalTok{ sd, }\AttributeTok{y =}\NormalTok{ median) }\SpecialCharTok{+}
  \FunctionTok{geom\_pointrange}\NormalTok{() }\SpecialCharTok{+}
  \CommentTok{\# add the line for the Intercept (Minimal Primary vegetation) at 0 (everything is compared to this)}
  \FunctionTok{geom\_hline}\NormalTok{(}\AttributeTok{yintercept =} \DecValTok{0}\NormalTok{, }\AttributeTok{size =} \FunctionTok{I}\NormalTok{(}\FloatTok{1.1}\NormalTok{), }\AttributeTok{color =} \FunctionTok{I}\NormalTok{(}\StringTok{"red"}\NormalTok{)) }\SpecialCharTok{+}
  \FunctionTok{coord\_flip}\NormalTok{() }\SpecialCharTok{+}
  \FunctionTok{theme\_bw}\NormalTok{() }\SpecialCharTok{+} \FunctionTok{labs}\NormalTok{(}\AttributeTok{x =} \StringTok{"Median Effect Estimate"}\NormalTok{, }\AttributeTok{y =} \StringTok{"Species Richness"}\NormalTok{)}
\end{Highlighting}
\end{Shaded}

\includegraphics{Analysis-for-insecta-data_files/figure-latex/unnamed-chunk-29-1.pdf}

\hypertarget{reference}{%
\subsection{Reference}\label{reference}}

De Palma, A., Sanchez-Ortiz, K. and Purvis, A. (2019) Calculating the
Biodiversity Intactness Index: the PREDICTS implementation. Zenodo. doi:
10.5281/ZENODO.3518067.

\end{document}
